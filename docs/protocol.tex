%% LyX 2.0.3 created this file.  For more info, see http://www.lyx.org/.
%% Do not edit unless you really know what you are doing.
\documentclass[12pt,english]{article}
\usepackage{mathptmx}
\renewcommand{\familydefault}{\rmdefault}
\usepackage[T1]{fontenc}
\usepackage[latin9]{inputenc}
\usepackage{geometry}
\geometry{verbose,tmargin=2cm,bmargin=2cm,lmargin=2cm,rmargin=2cm}

\makeatletter
%%%%%%%%%%%%%%%%%%%%%%%%%%%%%% Textclass specific LaTeX commands.
\newenvironment{lyxcode}
{\par\begin{list}{}{
\setlength{\rightmargin}{\leftmargin}
\setlength{\listparindent}{0pt}% needed for AMS classes
\raggedright
\setlength{\itemsep}{0pt}
\setlength{\parsep}{0pt}
\normalfont\ttfamily}%
 \item[]}
{\end{list}}

\makeatother

\usepackage{babel}
\begin{document}

\title{Alice Protocol}

\maketitle
\newpage{}

\tableofcontents{}

\newpage{}


\section{Introduction}

Alice acts as the connector between one or more NAME\_OF\_PRODUCT
clients, a PostgreSQL data store and one ore more PBX'es. Her main
task is to get and set data via a HTTP/JSON based protocol. This document
is the final authority on the Alice protocol.


\section{The Protocol}

When data is requested (get) the search parameters are always given
as a HTTP GET query string and the result is always returned as a
JSON document. When data is inserted/updated/deleted (set) the parameters
are always given as a HTTP POST request and the confirmation returned
from Alice will always be a JSON document.

In order to access any of the interfaces, a client must first authenticate
itself. This is done using OpenID. The login interface is the only
interface that can be accessed without prior login.

All data handled by the protocol is expected to be UTF-8 encoded.
Alice does not check whether this is in fact the case, so when implementing
clients, be sure to verify that all data sent to Alice are UTF-8 encoded.

Application level errors and exceptions are returned as JSON documents.
How these are handled is entirely up to the client. HTTP level errors
are returned as normal for HTTP servers.


\section{The Same Origin Policy - JSONP}

Same Origin Policy is a security feature found in the JavaScript implementation
in most browser. It prevents you from making requests to pages on
a different domain, another subdomain or through a different protocol.
The consequence of this is that you cannot query Alice from a client
on a different domain than the one used by Alice, which in some cases
can be a pain.

The solution to this problem is something called the JSONP standard.
This is supported transparently and out of the box by jQuery (and
probably other Javascript frameworks). You simply change the \emph{get()}
or \emph{post()} call to be JSONP based by stating that you would
like the return type to be \emph{json}.

If you prefer to build your own query strings by hand, then you must
add a \emph{jsoncallback} (preferred) or \emph{callback} (will work)
parameter with any value. The value of \emph{jsoncallback} is then
used by Alice to wrap the returned JSON string, like this:
\begin{lyxcode}
/get/queue\_length?jsoncallback=foo

foo(\{''length'':7\})
\end{lyxcode}
Strictly speaking this is only necessary if the client is running
in a browser and is being hosted from another domain than Alice.


\section{The Data Model}

Instead of using only a relational or document based model, we've
opted for a mixture. Data that is relational in nature is stored in
columns in a relational database, whereas everything else is stored
in a JSON document. When a client requests a specific type of data
Alice combines these two storage models into a final JSON document.

A benefit of this is that the client software easily can add new fields,
simply by expanding on the JSON document. If we take the \emph{/get/contact}
interface as an example, it might return a JSON looking like this:
\begin{lyxcode}
\{
\begin{lyxcode}
\textquotedbl{}name\textquotedbl{}:\textquotedbl{}Arthur~Dent\textquotedbl{},

``items'':{[}''towel'',''heart~of~gold'',''dressing~gown''{]},

``emailaddress'':''arthurdent42@somewhere.unknown.org'',

``cellphone'':''555-777-888-999'',

``dislikes'':''Vogons'',

``db\_columns'':
\begin{lyxcode}
\{
\begin{lyxcode}
\textquotedbl{}ce\_id\textquotedbl{}:1,

\textquotedbl{}ce\_name\textquotedbl{}:\textquotedbl{}Arthur~Dent\textquotedbl{},

``sip\_uri'':''sip://arthurdent.somewhere''
\end{lyxcode}
\}
\end{lyxcode}
\end{lyxcode}
\}
\end{lyxcode}
The \emph{db\_columns} JSON node is build from the actual columns
available in the relational database, in this case \emph{ce\_id},
\emph{ce\_name} and \emph{sip\_uri}. The data found in these columns
can be queried, sorted, grouped and/or joined by Alice. She is, so
to speak, aware of them. 

The other JSON nodes (\emph{name}, \emph{items}, \emph{emailaddress},
\emph{cellphone}, \emph{dislikes}) comes from the JSON document associated
with a contact entity. This is controlled 100\% by the client software.
Alice is completely oblivious as to what this document contains and
why. Things put here are only searchable by the client.

The strength of this model is that new fields can be added to a contact
entity on the fly, without needing any changes to Alice or the database.
Changes to Alice and/or the database itself is only necessary if the
new field must be searchable by Alice.

The weakness of this model is of course that a poorly designed/build
client can end up with some very bad JSON. We believe the strength
outweighs the weakness.


\section{Exceptions}

When bad things happen an exception is raised and an error JSON document
returned. These have the following format:
\begin{lyxcode}
\{
\begin{lyxcode}
\textquotedbl{}exception\_message\textquotedbl{}:\textquotedbl{}foo\textquotedbl{},~

\textquotedbl{}exception\textquotedbl{}:\textquotedbl{}SOME.EXCEPTION\textquotedbl{},~

\textquotedbl{}message\textquotedbl{}:\textquotedbl{}bar\textquotedbl{}
\end{lyxcode}
\}
\end{lyxcode}
All exceptions have this format. It is entirely up to the client to
decide what to do with an exception.


\section{Getting Data}


\subsection{/get/contact?ce\_id={[}natural integer{]}}

This interface returns the JSON document associated with the contact
entity identified by \emph{ce\_id}. Example JSON document:
\begin{lyxcode}
\{
\begin{lyxcode}
\textquotedbl{}name\textquotedbl{}:\textquotedbl{}Arthur~Dent\textquotedbl{},

``db\_columns'':
\begin{lyxcode}
\{
\begin{lyxcode}
\textquotedbl{}ce\_id\textquotedbl{}:1,

\textquotedbl{}ce\_name\textquotedbl{}:\textquotedbl{}Arthur~Dent\textquotedbl{},

``sip\_uri'':''sip://arthurdent.somewhere''
\end{lyxcode}
\}
\end{lyxcode}
\end{lyxcode}
\}
\end{lyxcode}
An empty JSON document is returned if \emph{ce\_id} doesn't exist
in the database:
\begin{lyxcode}
\{\}
\end{lyxcode}

\subsection{/get/contact\_attributes?ce\_id={[}natural integer{]}}

This interface returns the attributes associated with the \emph{ce\_id}
contact entity. Example JSON document:
\begin{lyxcode}
\{
\begin{lyxcode}
\textquotedbl{}attributes\textquotedbl{}:

{[}
\begin{lyxcode}
\{
\begin{lyxcode}
\textquotedbl{}phone\textquotedbl{}:\textquotedbl{}555-777-888\textquotedbl{},~

\textquotedbl{}db\_columns\textquotedbl{}:
\begin{lyxcode}
\{
\begin{lyxcode}
\textquotedbl{}ce\_id\textquotedbl{}:1,~

\textquotedbl{}org\_id\textquotedbl{}:1
\end{lyxcode}
\},~
\end{lyxcode}
\textquotedbl{}email\textquotedbl{}:\textquotedbl{}some@email.address\textquotedbl{}
\end{lyxcode}
\},

\{
\begin{lyxcode}
\textquotedbl{}phone\textquotedbl{}:\textquotedbl{}999-000-111\textquotedbl{},~

\textquotedbl{}db\_columns\textquotedbl{}:
\begin{lyxcode}
\{
\begin{lyxcode}
\textquotedbl{}ce\_id\textquotedbl{}:1,~

\textquotedbl{}org\_id\textquotedbl{}:2
\end{lyxcode}
\},~
\end{lyxcode}
\textquotedbl{}email\textquotedbl{}:\textquotedbl{}some.other@email.address\textquotedbl{}
\end{lyxcode}
\}
\end{lyxcode}
{]}
\end{lyxcode}
\}


\end{lyxcode}
Note that \emph{attributes} contains an array, meaning that one contact
entity can be associated with several attribute sets. This means that
the attributes of a contact entity can change depending on the context
in which it is seen.

If there are no attributes associated with a contact entity, the following
JSON document is returned:
\begin{lyxcode}
\{
\begin{lyxcode}
\textquotedbl{}attributes\textquotedbl{}:{[}{]}
\end{lyxcode}
\}


\end{lyxcode}

\subsection{/get/contact\_tags?ce\_id={[}natural integer{]}}

This interface returns the tags associated with the \emph{ce\_id}
contact entity. Example JSON document:
\begin{lyxcode}
\{
\begin{lyxcode}
\textquotedbl{}tags\textquotedbl{}:

{[}
\begin{lyxcode}
\{
\begin{lyxcode}
\textquotedbl{}db\_columns\textquotedbl{}:

\{
\begin{lyxcode}
\textquotedbl{}ce\_id\textquotedbl{}:1,~

\textquotedbl{}org\_id\textquotedbl{}:2
\end{lyxcode}
\},~

\textquotedbl{}tags\textquotedbl{}:

{[}
\begin{lyxcode}
\textquotedbl{}Ada\textquotedbl{},~

\textquotedbl{}Slackware\textquotedbl{},~

\textquotedbl{}Linux\textquotedbl{}
\end{lyxcode}
{]}
\end{lyxcode}
\}
\end{lyxcode}
{]}
\end{lyxcode}
\}
\end{lyxcode}
As with the attributes interface the \emph{tags} node contains an
array so several tag sets can be associated with a contact entity.

If there are no tags associated with a contact entity, the following
JSON document is returned:
\begin{lyxcode}
\{
\begin{lyxcode}
\textquotedbl{}tags\textquotedbl{}:{[}{]}
\end{lyxcode}
\}


\end{lyxcode}

\subsection{/get/organization?org\_id={[}natural\_integer{]}}

This interface returns the data associated with the \emph{org\_id}
organization. Example JSON:
\begin{lyxcode}
\{
\begin{lyxcode}
\textquotedbl{}name\textquotedbl{}:\textquotedbl{}AdaHeads~K/S\textquotedbl{},~

\textquotedbl{}db\_columns\textquotedbl{}:

\{
\begin{lyxcode}
\textquotedbl{}org\_name\textquotedbl{}:\textquotedbl{}AdaHeads~K/S\textquotedbl{},~

\textquotedbl{}identifier\textquotedbl{}:\textquotedbl{}sip://adaheads\textquotedbl{},~

\textquotedbl{}org\_id\textquotedbl{}:1
\end{lyxcode}
\}
\end{lyxcode}
\}
\end{lyxcode}
An empty JSON document is returned if no \emph{org\_id} organization
is found in the database:
\begin{lyxcode}
\{\}
\end{lyxcode}

\subsection{/get/org\_contacts?org\_id={[}natural integer{]}}

This interface returns all the contact entities associated with \emph{org\_id}.
Example JSON:
\begin{lyxcode}
\{
\begin{lyxcode}
\textquotedbl{}contacts\textquotedbl{}:

{[}
\begin{lyxcode}
\{
\begin{lyxcode}
\textquotedbl{}name\textquotedbl{}:\textquotedbl{}Zaphod~B.\textquotedbl{},~

\textquotedbl{}db\_columns\textquotedbl{}:

\{
\begin{lyxcode}
\textquotedbl{}ce\_id\textquotedbl{}:4,~

\textquotedbl{}org\_id\textquotedbl{}:2,~

\textquotedbl{}ce\_name\textquotedbl{}:\textquotedbl{}Zaphod~B.\textquotedbl{}
\end{lyxcode}
\}
\end{lyxcode}
\},~

\{
\begin{lyxcode}
\textquotedbl{}name\textquotedbl{}:\textquotedbl{}Tricia~Takanawa\textquotedbl{},~

\textquotedbl{}db\_columns\textquotedbl{}:

\{
\begin{lyxcode}
\textquotedbl{}ce\_id\textquotedbl{}:1,~

\textquotedbl{}org\_id\textquotedbl{}:2,~

\textquotedbl{}ce\_name\textquotedbl{}:\textquotedbl{}Tricia~Takanawa\textquotedbl{}
\end{lyxcode}
\}
\end{lyxcode}
\}
\end{lyxcode}
{]}
\end{lyxcode}
\}
\end{lyxcode}
The \emph{contacts} node contains an array, so it can hold multiple
contact entities. If no contact entities are associated with the given
\emph{org\_id} the following JSON document is returned:
\begin{lyxcode}
\{
\begin{lyxcode}
\textquotedbl{}contacts\textquotedbl{}:{[}{]}
\end{lyxcode}
\}
\end{lyxcode}

\subsection{/get/org\_contacts\_attributes?org\_id={[}natural integer{]}}

This interface returns all the contact entity attribute sets associated
with \emph{org\_id}. Example JSON document:
\begin{lyxcode}
\{
\begin{lyxcode}
\textquotedbl{}attributes\textquotedbl{}:

{[}
\begin{lyxcode}
\{
\begin{lyxcode}
\textquotedbl{}phone\textquotedbl{}:\textquotedbl{}555-777-888\textquotedbl{},~

\textquotedbl{}db\_columns\textquotedbl{}:
\begin{lyxcode}
\{
\begin{lyxcode}
\textquotedbl{}ce\_id\textquotedbl{}:1,~

\textquotedbl{}org\_id\textquotedbl{}:1
\end{lyxcode}
\},~
\end{lyxcode}
\textquotedbl{}email\textquotedbl{}:\textquotedbl{}some@email.address\textquotedbl{}
\end{lyxcode}
\},~

\{
\begin{lyxcode}
\textquotedbl{}phone\textquotedbl{}:\textquotedbl{}444-555-777\textquotedbl{},~

\textquotedbl{}db\_columns\textquotedbl{}:
\begin{lyxcode}
\{
\begin{lyxcode}
\textquotedbl{}ce\_id\textquotedbl{}:7,~

\textquotedbl{}org\_id\textquotedbl{}:1
\end{lyxcode}
\},~
\end{lyxcode}
\textquotedbl{}email\textquotedbl{}:\textquotedbl{}contact@me.here\textquotedbl{}
\end{lyxcode}
\}
\end{lyxcode}
{]}
\end{lyxcode}
\}
\end{lyxcode}
Because there can be several contact entity attribute sets associated
an \emph{org\_id} the \emph{attributes} JSON node contains an array.
If there are no attribute sets, the following JSON document is returned:
\begin{lyxcode}
\{
\begin{lyxcode}
\textquotedbl{}attributes\textquotedbl{}:{[}{]}
\end{lyxcode}
\}
\end{lyxcode}

\subsection{/get/org\_contacts\_tags?org\_id={[}natural integer{]}}

This interface returns all the contact entity tag sets associated
with \emph{org\_id}. Example JSON document:
\begin{lyxcode}
\{
\begin{lyxcode}
\textquotedbl{}tags\textquotedbl{}:

{[}
\begin{lyxcode}
\{
\begin{lyxcode}
\textquotedbl{}tags\textquotedbl{}:

{[}
\begin{lyxcode}
\textquotedbl{}Ada\textquotedbl{},~

\textquotedbl{}Slackware\textquotedbl{},~

\textquotedbl{}Linux\textquotedbl{}
\end{lyxcode}
{]},

\textquotedbl{}db\_columns\textquotedbl{}:
\begin{lyxcode}
\{
\begin{lyxcode}
\textquotedbl{}ce\_id\textquotedbl{}:1,~

\textquotedbl{}org\_id\textquotedbl{}:1
\end{lyxcode}
\}
\end{lyxcode}
\end{lyxcode}
\},~

\{
\begin{lyxcode}
\textquotedbl{}tags\textquotedbl{}:

{[}
\begin{lyxcode}
\textquotedbl{}Support\textquotedbl{},~

\textquotedbl{}Accounting\textquotedbl{},~

\textquotedbl{}Sales\textquotedbl{}
\end{lyxcode}
{]},

\textquotedbl{}db\_columns\textquotedbl{}:
\begin{lyxcode}
\{
\begin{lyxcode}
\textquotedbl{}ce\_id\textquotedbl{}:7,~

\textquotedbl{}org\_id\textquotedbl{}:1
\end{lyxcode}
\}
\end{lyxcode}
\end{lyxcode}
\}
\end{lyxcode}
{]}
\end{lyxcode}
\}
\end{lyxcode}
The \emph{tags} JSON node contains an array because there can be several
contact entity tag sets associated with an \emph{org\_id}. If there
are no tag sets, the following JSON document is returned:
\begin{lyxcode}
\{
\begin{lyxcode}
\textquotedbl{}tags\textquotedbl{}:{[}{]}
\end{lyxcode}
\}
\end{lyxcode}

\subsection{/get/queue}

This interface returns the current call queue. This is updated once
each second, so polling the /get/queue interface more often than once
each second is both wasteful and pointless. Calling for example:
\begin{lyxcode}
/get/queue
\end{lyxcode}
will return a JSON string like this:
\begin{lyxcode}
\{~~~~~
\begin{lyxcode}
\textquotedbl{}normal\textquotedbl{}:~
\end{lyxcode}
~~~~{[}~~~~~~~~~
\begin{lyxcode}
~~~~\{~~~~~~~~~~~~~
\begin{lyxcode}
~~~~\textquotedbl{}UTC\_start\_date\textquotedbl{}:\textquotedbl{}2012-02-22~14:23:30\textquotedbl{},

~~~~\textquotedbl{}id\textquotedbl{}:\textquotedbl{}GDhcf2VBww\textquotedbl{},~~~~~~~~~~~~~~~

~~~~\textquotedbl{}unix\_timestamp\textquotedbl{}:\textquotedbl{}1329920610\textquotedbl{},~~~~~~~~~~~~~

~~~~\textquotedbl{}callee\textquotedbl{}:5,~~~~~~~~~~~~~

~~~~\textquotedbl{}caller\textquotedbl{}:\textquotedbl{}d7sIp1kR\textquotedbl{}~~~~~~~~~
\end{lyxcode}
~~~~\}
\end{lyxcode}
~~~~{]},
\begin{lyxcode}
\textquotedbl{}high\textquotedbl{}:~
\end{lyxcode}
~~~~{[}~~~~~~~~~
\begin{lyxcode}
~~~~\{~~~~~~~~~~~~~~
\begin{lyxcode}
~~~~\textquotedbl{}UTC\_start\_date\textquotedbl{}:\textquotedbl{}2012-02-22~14:23:11\textquotedbl{},

~~~~\textquotedbl{}id\textquotedbl{}:\textquotedbl{}bRbYsMUVqx\textquotedbl{},~~~~~~~~~~~~~

~~~~\textquotedbl{}unix\_timestamp\textquotedbl{}:\textquotedbl{}1329920591\textquotedbl{},~~~~~~~~~~~~~~~~~

~~~~\textquotedbl{}callee\textquotedbl{}:3,~~~~~~~~~~~~~

~~~~\textquotedbl{}caller\textquotedbl{}:\textquotedbl{}oCgDF7ua\textquotedbl{}~~~~~~~~~
\end{lyxcode}
~~~~\}
\end{lyxcode}
~~~~{]},
\begin{lyxcode}
\textquotedbl{}low\textquotedbl{}:~
\end{lyxcode}
~~~~{[}~~~~~~~~~
\begin{lyxcode}
~~~~\{~~~~~~~~~~~~~
\begin{lyxcode}
~~~~\textquotedbl{}UTC\_start\_date\textquotedbl{}:\textquotedbl{}2012-02-22~14:23:17\textquotedbl{},~~~~~~~~~~~~
\begin{lyxcode}
~\textquotedbl{}id\textquotedbl{}:\textquotedbl{}VV8BFqGpqG\textquotedbl{},~~~~~~~~~~~~~
\end{lyxcode}
~~~~\textquotedbl{}unix\_timestamp\textquotedbl{}:\textquotedbl{}1329920597\textquotedbl{},~~~~~~~~~~~~~~

~~~~\textquotedbl{}callee\textquotedbl{}:9,~~~~~~~~~~~~~
\begin{lyxcode}
~\textquotedbl{}caller\textquotedbl{}:\textquotedbl{}ZQsRogwB\textquotedbl{}~~~~~~~~~
\end{lyxcode}
\end{lyxcode}
~~~~\}
\end{lyxcode}
~~~~{]},
\begin{lyxcode}
``length'':3
\end{lyxcode}
\}
\end{lyxcode}
The \emph{normal}, \emph{low} and \emph{high} nodes represent priority
in the queue. All these naturally contains arrays of calls. The \emph{length}
node contains the totalt length of the queue. The\emph{ callee} node
maps to an \emph{org\_id}. An empty queue returns the following JSON
document:
\begin{lyxcode}
\{
\begin{lyxcode}
\textquotedbl{}normal\textquotedbl{}:{[}{]},~

\textquotedbl{}length\textquotedbl{}:0,~

\textquotedbl{}high\textquotedbl{}:{[}{]},~

\textquotedbl{}low\textquotedbl{}:{[}{]}
\end{lyxcode}
\}
\end{lyxcode}

\subsection{/get/queue\_length}

This interface returns the length of the current call queue. Example
JSON document:
\begin{lyxcode}
\{
\begin{lyxcode}
\textquotedbl{}length\textquotedbl{}:7
\end{lyxcode}
\}
\end{lyxcode}

\end{document}
