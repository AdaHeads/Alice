%% LyX 2.0.0 created this file.  For more info, see http://www.lyx.org/.
%% Do not edit unless you really know what you are doing.
\documentclass[12pt,english]{article}
\usepackage[T1]{fontenc}
\usepackage[latin9]{inputenc}
\usepackage{geometry}
\geometry{verbose,tmargin=2cm,bmargin=2cm,lmargin=2cm,rmargin=2cm}

\makeatletter
%%%%%%%%%%%%%%%%%%%%%%%%%%%%%% Textclass specific LaTeX commands.
\newenvironment{lyxcode}
{\par\begin{list}{}{
\setlength{\rightmargin}{\leftmargin}
\setlength{\listparindent}{0pt}% needed for AMS classes
\raggedright
\setlength{\itemsep}{0pt}
\setlength{\parsep}{0pt}
\normalfont\ttfamily}%
 \item[]}
{\end{list}}

\makeatother

\usepackage{babel}
\begin{document}

\title{Alice Protocol}

\maketitle
\newpage{}

\tableofcontents{}

\newpage{}


\section{Introduction}

Alice acts as the connector between one or more NAME\_OF\_PRODUCT
clients and a data store/soft PBX that is comprised of a mixture of
LDAP/PostgreSQL servers and one ore more FreeSWITCH servers. Her main
task is to get and set data via a HTTP/JSON based protocol. This document
is the final authority on the protocol understood by Alice.


\section{The Protocol}

In its most basic form the protocol consists of a set of {}``get''
and {}``set'' interfaces. You can either get some data from Alice,
or you can tell her to set some data. 

When data is requested (get) the search parameters are always given
as a HTTP GET query string and the result is always returned as a
JSON string. When data is added/updated/deleted (set) the parameters
are always given as a HTTP POST request and the confirmation returned
from Alice will always be a JSON string.

In order to access any of the interfaces, a client must first authenticate
itself. This is done using OpenID. The login interface is the only
interface that can be accessed without prior login.

All data handled by the protocol is expected to be UTF-8 encoded.
Alice does not check whether this is in fact the case, so when implementing
clients, be sure to verify that all data sent to Alice is in fact
UTF-8 encoded.

All interfaces should of course be prefixed with whatever server:port
is used for a given Alice instance.


\section{Getting Data}


\subsection{/get/company?o={[}COMPANY{]}}

This interface return all the data associated with COMPANY. If the
given company doesn't exist in the database, then an empty JSON string
(\{\}) is returned.

Only company data is returned when calling this interface. No employee
data is returned. Calling for example
\begin{lyxcode}
/get/company?o=Hansen\%20VVS
\end{lyxcode}
will return a JSON string similar to this:
\begin{lyxcode}
\{

\textquotedbl{}o=Hansen~VVS,dc=example,dc=com\textquotedbl{}:~

~~~~\{

~~~~\textquotedbl{}l\textquotedbl{}:~\textquotedbl{}N�stved\textquotedbl{},

~~~~\textquotedbl{}o\textquotedbl{}:~\textquotedbl{}Hansen~VVS\textquotedbl{},

~~~~\textquotedbl{}street\textquotedbl{}:~\textquotedbl{}Slagelsevej~13\textquotedbl{},

~~~~\textquotedbl{}objectClass\textquotedbl{}:

~~~~~~~~{[}

~~~~~~~~\textquotedbl{}top\textquotedbl{},~~~~~~~~~~~~~

~~~~~~~~\textquotedbl{}organization\textquotedbl{}

~~~~~~~~{]},~~~~~~~~~

~~~~\textquotedbl{}telephoneNumber\textquotedbl{}:~\textquotedbl{}+45~5544~3321\textquotedbl{},

~~~~\textquotedbl{}postalCode\textquotedbl{}:~\textquotedbl{}DK-4700\textquotedbl{}

~~~~\}

\}
\end{lyxcode}
TODO: Document each JSON field.


\subsection{/get/person?o={[}COMPANY{]}\&cn={[}PERSON{]}}

This interface return the data associated with PERSON employed by
COMPANY. Calling for example
\begin{lyxcode}
/get/person?o=Hansen\%20VVS\&cn=Brian\%20Hansen
\end{lyxcode}
will return a JSON string similar to this:
\begin{lyxcode}
\{~~~~~

\textquotedbl{}cn=Brian~Hansen,o=HansenVVS,dc=example,dc=com\textquotedbl{}:~

~~~~\{~~~~~~~~~

~~~~\textquotedbl{}cn\textquotedbl{}:~\textquotedbl{}Brian~Hansen\textquotedbl{},~~~~~~~~~

~~~~\textquotedbl{}objectClass\textquotedbl{}:~

~~~~~~~~{[}~~~~~~~~~~~~~

~~~~~~~~\textquotedbl{}person\textquotedbl{},~~~~~~~~~~~~~

~~~~~~~~\textquotedbl{}organizationalPerson\textquotedbl{},~~~~~~~~~~~~~

~~~~~~~~\textquotedbl{}top\textquotedbl{}~~~~~~~~~

~~~~~~~~{]},~~~~~~~~~

~~~~\textquotedbl{}telephoneNumber\textquotedbl{}:~\textquotedbl{}+45~2266~1155\textquotedbl{},

~~~~\textquotedbl{}sn\textquotedbl{}:~\textquotedbl{}Hansen\textquotedbl{},~~~~~~~~~

~~~~\textquotedbl{}title\textquotedbl{}:~\textquotedbl{}Gasmester\textquotedbl{}~~~~~

~~~~\}~

\}
\end{lyxcode}
TODO: Document each JSON field.


\subsection{/get/persons?o={[}COMPANY{]}}

This interface return the persons associated with COMPANY. Calling
for example
\begin{lyxcode}
/get/persons?o=Hansen\%20VVS
\end{lyxcode}
will return a JSON string similar to this:
\begin{lyxcode}
\{~~~~~

\textquotedbl{}cn=Ole~Johnsen,o=Hansen~VVS,dc=example,dc=com\textquotedbl{}:~

~~~~\{~~~~~~~~~

~~~~\textquotedbl{}cn\textquotedbl{}:~\textquotedbl{}Ole~Kappel~Johnsen\textquotedbl{},~~~~~~~~~

~~~~\textquotedbl{}objectClass\textquotedbl{}:~

~~~~~~~~{[}~~~~~~~~~~~~~

~~~~~~~~\textquotedbl{}person\textquotedbl{},~~~~~~~~~~~~~

~~~~~~~~\textquotedbl{}organizationalPerson\textquotedbl{},~~~~~~~~~~~~~

~~~~~~~~\textquotedbl{}top\textquotedbl{}~~~~~~~~~

~~~~~~~~{]},~~~~~~~~~

~~~~\textquotedbl{}telephoneNumber\textquotedbl{}:~\textquotedbl{}+45~55~11~66~35\textquotedbl{},~~~~~~~~~

~~~~\textquotedbl{}sn\textquotedbl{}:~\textquotedbl{}Johnsen\textquotedbl{},~~~~~~~~~

~~~~\textquotedbl{}title\textquotedbl{}:~\textquotedbl{}Svend\textquotedbl{}~~~~~

~~~~\},

\textquotedbl{}cn=Brian~Hansen,o=Hansen~VVS,dc=example,dc=com\textquotedbl{}:~

~~~~\{~~~~~~~~~

~~~~\textquotedbl{}cn\textquotedbl{}:~\textquotedbl{}Brian~Hansen\textquotedbl{},~~~~~~~~~

~~~~\textquotedbl{}objectClass\textquotedbl{}:~

~~~~~~~~{[}~~~~~~~~~~~~~

~~~~~~~~\textquotedbl{}person\textquotedbl{},~~~~~~~~~~~~~

~~~~~~~~\textquotedbl{}organizationalPerson\textquotedbl{},~~~~~~~~~~~~~

~~~~~~~~\textquotedbl{}top\textquotedbl{}~~~~~~~~~

~~~~~~~~{]},~~~~~~~~~

~~~~\textquotedbl{}telephoneNumber\textquotedbl{}:~\textquotedbl{}+45~2266~1155\textquotedbl{},~~~~~~~~~

~~~~\textquotedbl{}sn\textquotedbl{}:~\textquotedbl{}Hansen\textquotedbl{},~~~~~~~~~

~~~~\textquotedbl{}title\textquotedbl{}:~\textquotedbl{}Gasmester\textquotedbl{}~~~~~

~~~~\}~

\}
\end{lyxcode}
TODO: Document each JSON field.


\subsection{/get/queue}

This interface return the current call queue. This is updated once
each second, so polling the /get/queue interface more often than once
each second is both wasteful and pointless. Calling for example:
\begin{lyxcode}
/get/queue
\end{lyxcode}
will return a JSON string like this:
\begin{lyxcode}
\{~~~~~

\textquotedbl{}normal\textquotedbl{}:~

~~~~{[}~~~~~~~~~

~~~~\{~~~~~~~~~~~~~

~~~~\textquotedbl{}UTC\_start\_date\textquotedbl{}:~\textquotedbl{}2012-02-22~14:23:30\textquotedbl{},

~~~~\textquotedbl{}id\textquotedbl{}:~\textquotedbl{}GDhcf2VBww\textquotedbl{},~~~~~~~~~~~~~~~

~~~~\textquotedbl{}unix\_timestamp\textquotedbl{}:~\textquotedbl{}1329920610\textquotedbl{},~~~~~~~~~~~~~

~~~~\textquotedbl{}callee\textquotedbl{}:~\textquotedbl{}mU1Ff16h\textquotedbl{},~~~~~~~~~~~~~

~~~~\textquotedbl{}caller\textquotedbl{}:~\textquotedbl{}d7sIp1kR\textquotedbl{}~~~~~~~~~

~~~~\}

~~~~{]},

\textquotedbl{}high\textquotedbl{}:~

~~~~{[}~~~~~~~~~

~~~~\{~~~~~~~~~~~~~~

~~~~\textquotedbl{}UTC\_start\_date\textquotedbl{}:~\textquotedbl{}2012-02-22~14:23:11\textquotedbl{},

~~~~\textquotedbl{}id\textquotedbl{}:~\textquotedbl{}bRbYsMUVqx\textquotedbl{},~~~~~~~~~~~~~

~~~~\textquotedbl{}unix\_timestamp\textquotedbl{}:~\textquotedbl{}1329920591\textquotedbl{},~~~~~~~~~~~~~~~~~

~~~~\textquotedbl{}callee\textquotedbl{}:~\textquotedbl{}pMT3k2fb\textquotedbl{},~~~~~~~~~~~~~

~~~~\textquotedbl{}caller\textquotedbl{}:~\textquotedbl{}oCgDF7ua\textquotedbl{}~~~~~~~~~

~~~~\}

~~~~{]},

\textquotedbl{}low\textquotedbl{}:~

~~~~{[}~~~~~~~~~

~~~~\{~~~~~~~~~~~~~

~~~~\textquotedbl{}UTC\_start\_date\textquotedbl{}:~\textquotedbl{}2012-02-22~14:23:17\textquotedbl{},~~~~~~~~~~~~~\textquotedbl{}id\textquotedbl{}:~\textquotedbl{}VV8BFqGpqG\textquotedbl{},~~~~~~~~~~~~~

~~~~\textquotedbl{}unix\_timestamp\textquotedbl{}:~\textquotedbl{}1329920597\textquotedbl{},~~~~~~~~~~~~~~

~~~~\textquotedbl{}callee\textquotedbl{}:~\textquotedbl{}7ar4Y1qc\textquotedbl{},~~~~~~~~~~~~~\textquotedbl{}caller\textquotedbl{}:~

~~~~\textquotedbl{}ZQsRogwB\textquotedbl{}~~~~~~~~~

~~~~\}

~~~~{]}

\}
\end{lyxcode}
TODO: Document each JSON field.


\subsection{/get/queue?kind=length}

This interface returns the length of the current call queue. Calling
\begin{lyxcode}
/get/queue?kind=length
\end{lyxcode}
will return a JSON string like this:
\begin{lyxcode}
\{

\textquotedbl{}length\textquotedbl{}:\textquotedbl{}7\textquotedbl{}

\}
\end{lyxcode}
TODO: Document each JSON field.
\end{document}
